\documentclass{article}
\usepackage{fancyhdr}
\usepackage{amsmath}
\pagestyle{fancy}
\lhead{INF01120\\Resumo: Software Quality}
\rhead{00324422\\Erick Larratéa Knoblich}

\begin{document}

\begin{center}

    \vspace*{-8mm}
    \textbf{\large{Resumo: Software Quality}}
    \vspace*{-4mm}

\end{center}

\noindent\rule{\textwidth}{0.5pt}

\bigskip

O texto aborda características que fazem um software de qualidade, não só aquelas as quais os usuários percebem, as externas, mas também aquelas que só os programadores percebem, as internas. Elas diferenciam um programa amador que só será feito uma vez e usado para um só caso de um software profissional que será editado diversas vezes ao longo dos anos por diferentes pessoas de diferentes áreas, cada vaz mais com novas adaptações e extensões para outros fins. Para isso, precisamos levar em conta o conteúdo interno para termos um código excelente e a experiência do usuário para termos idea do que disponibilizar para o consumo, afinal usuários são imprevisíveis e não são programadores. Essencialmente, o software precisa ser elegatemente arquitetado e ser facilmente utilzado por qualquer pessoa de qualquer área, principalmente de áreas totalmente diferentes da área de computação.

\bigskip

\noindent\rule{\textwidth}{0.5pt}

\bigskip

\textbf{Corretude:}

\medskip

A qualidade de corretude é a habilidade de um software realizar corretamente aquilo que foi feito para fazer, como realizar qualquer algoritmo sem erro, mas não necessariamente de uma maneira elegante, ou seja, corretude é o mínimo necessário para qualquer aplicação.

\medskip

\textbf{Robustez:}

\medskip

A qualidade de robustez é a habilidade de um software reagir inteligentemente em situações não programadas ou induzidas. Por exemplo, ao recebermos um input de um usuário, precisamos ter certeza de que esse input é válido, senão podemos ter problemas no código. 

\medskip

\textbf{Extensibilidade:}

\medskip

A qualidade de extensibilidade é a habilidade de adaptar o software para a realização de tarefas diferentas daquelas originalmente especificadas, como realizar um algoritmo de ordenação de uma maneira diferente, mas semelhante, como usar um outro elemento no vetor como elemento pivô.

\medskip

\textbf{Reusabilidade:}

\medskip

A qualidade de reusabilidade é a habilidade de reutilizar aquilo que foi feito no software para outros meios, como a reutilização de classes por meio de classes herdadas e, com isso, a reutilização de propriedades e métodos, além de reescrita caso necessário.

\medskip

\textbf{Compatibilidade:}

\medskip

A qualidade de compatibilidade é a habilidade de combinar certos aspectos de um software com aspectos de outros, como não assumir que um formato de data e hora ou de endereço específico de uma região do mundo serão sempre usados, pois o software deve ser compatível para todos os formatos que serão usados por outras pessoas, assim como formatos de arquivos mais usados em diferentes regiões do mundo.

\pagebreak

\begin{center}

    \vspace*{-7mm}
    \textbf{\large{Resumo: Software Quality}}
    \vspace*{-5mm}

\end{center}

\noindent\rule{\textwidth}{0.5pt}

\bigskip

\textbf{Eficiência:}

\medskip

A qualidade de eficiência é a habilidade de um software realizar aquilo que foi feito para fazer de maneira elegante, eficiente, ou seja, usar e ocupar o mínimo de memória, ser rápido, ser legível, etc.

\medskip

\textbf{Portabilidade:}

\medskip

A qualidade de portabilidade é a habilidade de um software funcionar em máquinas e ambientes diferentes, como software criados em linguagens de programação multiplataforma como Java e Flutter.

\medskip

\textbf{Facilidade de Uso:}

\medskip

A qualidade de facilidade de uso é a habilidade de diferentes pessoas de diferentes áreas, muitas vezes pessoas que nem sabem programar, conseguirem usar o software com facilidade e conseguirem resolver os problemas que precisam, como uma interface amigável, direta e organizada.

\end{document}