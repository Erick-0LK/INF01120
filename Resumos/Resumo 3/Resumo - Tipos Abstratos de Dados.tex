\documentclass{article}
\usepackage{fancyhdr}
\usepackage{amsmath}
\pagestyle{fancy}
\lhead{INF01120\\Resumo: Tipos Abstratos de Dados}
\rhead{00324422\\Erick Larratéa Knoblich}

\begin{document}

\begin{center}

    \vspace*{-7mm}
    \textbf{\large{Resumo: Tipos Abstratos de Dados}}
    \vspace*{-4mm}

\end{center}

\noindent\rule{\textwidth}{0.5pt}

\bigskip

O texto aborda tipos abstratos de dados, que são, explicando de uma forma inicialmente mais simples, uma forma de representar dados complexos numa abstração de alto nível amigável ao usuário por meio de funções tal que ele não precise entender como, de fato, o algoritmo funciona ou ter acesso ao algoritmo em si, uma vez que as funções fornecidas são o suficiente para utilizar o tipo abstrato de dado. Para entendermos melhor o que um tipo abstrato de dado é, revisaremos tipos de dados primitivos e tipos de dados compostos.

\bigskip

\noindent\rule{\textwidth}{0.5pt}

\bigskip

\textbf{Tipos de Dados Primitivos:}

\medskip

Tipos de dados primitivos estão presentes em todas as linguagens de programação e representam conceitos básicos, como os primitivos numérios inteiros, floats, longs, shorts e primitivos textuais como caracteres. Tipos de dados de maior complexidade e de maior alto nível utilizam esses tipos de dados.

\medskip

\textbf{Tipos de Dados Compostos:}

\medskip

Tipos de dados compostos são uma junção de tipos de dados primitivos que formam um dado mais complexo, mas não fornecem uma abstração de alto nível por meio de funções assim como os tipos de dados abstratos fornecem, como estruturas em C, como uma estrutura ponto que possui dois tipos de dados primitivos: um inteiro para a posição $x$ e outro inteiro para a posição $y$.

\medskip

\textbf{Tipos de Dados Abstratos:}

\medskip

Tipos de dados abstratos são dados complexos, como classes e objetos, cujo comportamento é definido por um conjunto de valores e um conjunto de operações, propriedades e métodos no contexto classe-objeto. Por ser abstrato e de alto nível, apenas as operações as quais devem ser executadas são mostradas, ou seja, não é mostrado como essas operações serão implementadas. Um exemplo de um tipo abstrato de dado é um objeto pilha, instância da classe pilha, que possui métodos bem definidos para operações com suas propriedades, como inicialização, inserção, remoção e duas checagens para saber se a pilha está cheia ou vazia. A classe, o objeto e os métodos, bem definidos e claros, formam um dado complexo de alto nível que pode ser usado pelo usuário sem esse saber como as operações sobre a pilha funcionam, isto é, um tipo abstrato de dado.

\end{document}